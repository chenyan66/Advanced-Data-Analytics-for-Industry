% Options for packages loaded elsewhere
\PassOptionsToPackage{unicode}{hyperref}
\PassOptionsToPackage{hyphens}{url}
%
\documentclass[
]{article}
\usepackage{amsmath,amssymb}
\usepackage{iftex}
\ifPDFTeX
  \usepackage[T1]{fontenc}
  \usepackage[utf8]{inputenc}
  \usepackage{textcomp} % provide euro and other symbols
\else % if luatex or xetex
  \usepackage{unicode-math} % this also loads fontspec
  \defaultfontfeatures{Scale=MatchLowercase}
  \defaultfontfeatures[\rmfamily]{Ligatures=TeX,Scale=1}
\fi
\usepackage{lmodern}
\ifPDFTeX\else
  % xetex/luatex font selection
\fi
% Use upquote if available, for straight quotes in verbatim environments
\IfFileExists{upquote.sty}{\usepackage{upquote}}{}
\IfFileExists{microtype.sty}{% use microtype if available
  \usepackage[]{microtype}
  \UseMicrotypeSet[protrusion]{basicmath} % disable protrusion for tt fonts
}{}
\makeatletter
\@ifundefined{KOMAClassName}{% if non-KOMA class
  \IfFileExists{parskip.sty}{%
    \usepackage{parskip}
  }{% else
    \setlength{\parindent}{0pt}
    \setlength{\parskip}{6pt plus 2pt minus 1pt}}
}{% if KOMA class
  \KOMAoptions{parskip=half}}
\makeatother
\usepackage{xcolor}
\usepackage[margin=1in]{geometry}
\usepackage{graphicx}
\makeatletter
\def\maxwidth{\ifdim\Gin@nat@width>\linewidth\linewidth\else\Gin@nat@width\fi}
\def\maxheight{\ifdim\Gin@nat@height>\textheight\textheight\else\Gin@nat@height\fi}
\makeatother
% Scale images if necessary, so that they will not overflow the page
% margins by default, and it is still possible to overwrite the defaults
% using explicit options in \includegraphics[width, height, ...]{}
\setkeys{Gin}{width=\maxwidth,height=\maxheight,keepaspectratio}
% Set default figure placement to htbp
\makeatletter
\def\fps@figure{htbp}
\makeatother
\setlength{\emergencystretch}{3em} % prevent overfull lines
\providecommand{\tightlist}{%
  \setlength{\itemsep}{0pt}\setlength{\parskip}{0pt}}
\setcounter{secnumdepth}{-\maxdimen} % remove section numbering
\ifLuaTeX
  \usepackage{selnolig}  % disable illegal ligatures
\fi
\IfFileExists{bookmark.sty}{\usepackage{bookmark}}{\usepackage{hyperref}}
\IfFileExists{xurl.sty}{\usepackage{xurl}}{} % add URL line breaks if available
\urlstyle{same}
\hypersetup{
  pdftitle={HW7},
  pdfauthor={Chenyan Feng},
  hidelinks,
  pdfcreator={LaTeX via pandoc}}

\title{HW7}
\author{Chenyan Feng}
\date{2023-03-09}

\begin{document}
\maketitle

\hypertarget{r-markdown}{%
\subsection{R Markdown}\label{r-markdown}}

This is an R Markdown document. Markdown is a simple formatting syntax
for authoring HTML, PDF, and MS Word documents. For more details on
using R Markdown see \url{http://rmarkdown.rstudio.com}.

When you click the \textbf{Knit} button a document will be generated
that includes both content as well as the output of any embedded R code
chunks within the document. You can embed an R code chunk like this:

\#p1 \#(a) library(ISLR) data(Default) names(Default) summary(Default)
set.seed(1) LG.fit = glm(default\textasciitilde income+balance,
family=binomial) summary(LG.fit)

\#(b) train\_set = sample(dim(Default){[}1{]},dim(Default){[}1{]}/1.5)
\#i. Split the sample set into a training set and a validation set.
NLG.fit = glm(default \textasciitilde{} income + balance, data =
Default, family = ``binomial'', subset = train\_set) \#ii. Fit a
logistic regression model using only the training data set. probs =
predict(NLG.fit, newdata = Default{[}-train\_set, {]}, type =
``response'') pred.glm = rep(``No'', length(probs)) pred.glm{[}probs
\textgreater{} 0.5{]} = ``Yes'' \#iii. Obtain a prediction of default
status for each individual in the validation set using a threshold of
0.5. mean(pred.glm != Default{[}-train\_set, {]}\$default)

\#(c) train\_set = sample(dim(Default){[}1{]},dim(Default){[}1{]}/1.5)
NLG.fit = glm(default \textasciitilde{} income + balance, data =
Default, family = ``binomial'', subset = train\_set) probs =
predict(NLG.fit, newdata = Default{[}-train\_set, {]}, type =
``response'') pred.glm = rep(``No'', length(probs)) pred.glm{[}probs
\textgreater{} 0.5{]} = ``Yes'' mean(pred.glm != Default{[}-train\_set,
{]}\$default)

train\_set = sample(dim(Default){[}1{]},dim(Default){[}1{]}/1.5) NLG.fit
= glm(default \textasciitilde{} income + balance, data = Default, family
= ``binomial'', subset = train\_set) probs = predict(NLG.fit, newdata =
Default{[}-train\_set, {]}, type = ``response'') pred.glm = rep(``No'',
length(probs)) pred.glm{[}probs \textgreater{} 0.5{]} = ``Yes''
mean(pred.glm != Default{[}-train\_set, {]}\$default)

train\_set = sample(dim(Default){[}1{]},dim(Default){[}1{]}/1.5) NLG.fit
= glm(default \textasciitilde{} income + balance, data = Default, family
= ``binomial'', subset = train\_set) probs = predict(NLG.fit, newdata =
Default{[}-train\_set, {]}, type = ``response'') pred.glm = rep(``No'',
length(probs)) pred.glm{[}probs \textgreater{} 0.5{]} = ``Yes''
mean(pred.glm != Default{[}-train\_set, {]}\$default)

\#(d)adding one student variable, check the test error train\_set
\textless- sample(dim(Default){[}1{]}, dim(Default){[}1{]} / 1.5) LG.glm
\textless- glm(default \textasciitilde{} income + balance + student,
data = Default, family = ``binomial'', subset = train\_set) pred.glm
\textless- rep(``No'', length(probs)) probs \textless- predict(LG.glm,
newdata = Default{[}-train\_set, {]}, type = ``response'')
pred.glm{[}probs \textgreater{} 0.5{]} \textless- ``Yes'' mean(pred.glm
!= Default{[}-train\_set, {]}\$default)

\#p2 \#(a) Generate a simulated data set as follows: x=rnorm(200)
y=x-2*x\^{}2+rnorm(200) \#(b) plot(x,y) \#(c) library(boot) Data =
data.frame(x,y) set.seed(1) LG.fit = glm(y\textasciitilde x)
cv.glm(Data, LG.fit)\$delta

LG2.fit = glm(y\textasciitilde poly(x,2)) cv.glm(Data, LG2.fit)\$delta

LG3.fit = glm(y\textasciitilde poly(x,3)) cv.glm(Data, LG3.fit)\$delta

LG4.fit = glm(y\textasciitilde poly(x,4)) cv.glm(Data, LG4.fit)\$delta

\#(d) set.seed(3) LG.fit = glm(y\textasciitilde x) cv.glm(Data,
LG.fit)\$delta

LG.fit = glm(y\textasciitilde poly(x,2)) cv.glm(Data, LG.fit)\$delta

LG.fit = glm(y\textasciitilde poly(x,3)) cv.glm(Data, LG.fit)\$delta

LG.fit = glm(y\textasciitilde poly(x,4)) cv.glm(Data, LG.fit)\$delta

\#(f) \# Generate simulated data set set.seed(1) x \textless- rnorm(200)
y \textless- x - 2*x\^{}2 + rnorm(200)

\hypertarget{define-the-four-models}{%
\section{Define the four models}\label{define-the-four-models}}

model1 \textless- lm(y \textasciitilde{} x) model2 \textless- lm(y
\textasciitilde{} x + I(x\^{}2)) model3 \textless- lm(y
\textasciitilde{} x + I(x\^{}2) + I(x\^{}3)) model4 \textless- lm(y
\textasciitilde{} x + I(x\^{}2) + I(x\^{}3) + I(x\^{}4))

\hypertarget{create-a-function-to-perform-5-fold-cross-validation-and-return-the-cross-validation-error}{%
\section{Create a function to perform 5-fold cross-validation and return
the cross-validation
error}\label{create-a-function-to-perform-5-fold-cross-validation-and-return-the-cross-validation-error}}

cv \textless- function(model) \{ cv.error \textless- rep(NA, 5) folds
\textless- cut(seq\_along(x), breaks = 5, labels = FALSE) for (i in 1:5)
\{ test.index \textless- which(folds == i) train.index \textless-
which(folds != i) cv.error{[}i{]} \textless- mean((y{[}test.index{]} -
predict(model, newdata = data.frame(x = x{[}test.index{]})))\^{}2) \}
mean(cv.error) \}

\hypertarget{compute-the-cross-validation-errors-for-the-four-models}{%
\section{Compute the cross-validation errors for the four
models}\label{compute-the-cross-validation-errors-for-the-four-models}}

cv.error1 \textless- cv(model1) cv.error2 \textless- cv(model2)
cv.error3 \textless- cv(model3) cv.error4 \textless- cv(model4)

\hypertarget{print-the-cross-validation-errors}{%
\section{Print the cross-validation
errors}\label{print-the-cross-validation-errors}}

cat(``CV error for model 1:'', cv.error1, ``\n'') cat(``CV error for
model 2:'', cv.error2, ``\n'') cat(``CV error for model 3:'', cv.error3,
``\n'') cat(``CV error for model 4:'', cv.error4, ``\n'')

\#(g) set.seed(3) x \textless- rnorm(200) y \textless- x - 2*x\^{}2 +
rnorm(200)

\hypertarget{define-the-four-models-1}{%
\section{Define the four models}\label{define-the-four-models-1}}

model1\_10 \textless- lm(y \textasciitilde{} x) model2\_10 \textless-
lm(y \textasciitilde{} x + I(x\^{}2)) model3\_10 \textless- lm(y
\textasciitilde{} x + I(x\^{}2) + I(x\^{}3)) model4\_10 \textless- lm(y
\textasciitilde{} x + I(x\^{}2) + I(x\^{}3) + I(x\^{}4))

\hypertarget{create-a-function-to-perform-5-fold-cross-validation-and-return-the-cross-validation-error-1}{%
\section{Create a function to perform 5-fold cross-validation and return
the cross-validation
error}\label{create-a-function-to-perform-5-fold-cross-validation-and-return-the-cross-validation-error-1}}

cv \textless- function(model) \{ cv.error \textless- rep(NA, 10) folds
\textless- cut(seq\_along(x), breaks = 10, labels = FALSE) for (i in
1:10) \{ test.index \textless- which(folds == i) train.index \textless-
which(folds != i) cv.error{[}i{]} \textless- mean((y{[}test.index{]} -
predict(model, newdata = data.frame(x = x{[}test.index{]})))\^{}2) \}
mean(cv.error) \}

\hypertarget{compute-the-cross-validation-errors-for-the-four-models-1}{%
\section{Compute the cross-validation errors for the four
models}\label{compute-the-cross-validation-errors-for-the-four-models-1}}

cv.error1\_10 \textless- cv(model1\_10) cv.error2\_10 \textless-
cv(model2\_10) cv.error3\_10 \textless- cv(model3\_10) cv.error4\_10
\textless- cv(model4\_10)

\hypertarget{print-the-cross-validation-errors-1}{%
\section{Print the cross-validation
errors}\label{print-the-cross-validation-errors-1}}

cat(``CV error for model 1:'', cv.error1\_10, ``\n'') cat(``CV error for
model 2:'', cv.error2\_10, ``\n'') cat(``CV error for model 3:'',
cv.error3\_10, ``\n'') cat(``CV error for model 4:'', cv.error4\_10,
``\n'')

\hypertarget{including-plots}{%
\subsection{Including Plots}\label{including-plots}}

You can also embed plots, for example:

\includegraphics{hw7_files/figure-latex/pressure-1.pdf}

Note that the \texttt{echo\ =\ FALSE} parameter was added to the code
chunk to prevent printing of the R code that generated the plot.

\end{document}
